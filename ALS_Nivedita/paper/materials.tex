\section{Materials}
\label{sec:materials}
This part will concentrate on the description of the dataset used in our project. The dataset refers to a study on Amyotrophic Lateral Sclerosis (ALS) leaded by Nivedita Agarwal. The data is recorded with a 3T scanner (unknown brand) at Utah Brain Institute. Patients are referred as 1** while controls with 2** notation. The dataset originally includes the recordings of 12 patients and 12 controls. However, in practice the object 108 and the control 211 are missed. 

The dataset includes the following kind of data:
\begin{enumerate}
\item anatomical scan (1x1x1)
\item diffusion recording 
	\begin{enumerate}
	\item 64 gradients $+ B0$ (total 65 directions)
	\item bval=1000.
	\end{enumerate}	 
\end{enumerate}

Although there are many subfolders inside, but the only two subfolders of interest are MP\underline{ }RAGE\underline{ }1X1X1\underline{ }ND\underline{ }0003 (anatomy) and 
\\DIFF2DEPI\underline{ }EKJ(64dirs)\underline{ }14 (diffusion data). 
%The table of gradients can be found in the table \ref{table:gradient}

Recently, Nivedita Agarwal provides three additional subjects 111, 112 and 113. Note that these three additional subjects 111, 112 and 113 have 3 series of data, and the MP\underline{ }RAGE\underline{ }1X1X1\underline{ }ND\underline{ }0003 subfolder includes 672 images, and DIFF2DEPI\underline{ }EKJ(64dirs)\underline{ }14 has 195 images rather than 224 and 65. After separating each series, we can re-construct the tractography of the whole brain.
 
