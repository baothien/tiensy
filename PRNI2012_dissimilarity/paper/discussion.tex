\section{Discussion}
\label{sec:discussion}
In this document we investigated the degree of approximation of the
dissimilarity representation for the goal of preserving the relative
distances between streamlines within tractographies. Empirical
assessment has been conducted on two different datasets and through
various prototype selection methods. All of the results from both
simulated data and real tractography data reached correlation $\ge
0.95$ with respect to the distances in the original space. This fact
proved that the dissimilarity representation works well for preserving
the relative distances. Moreover on tractography data the maximum correlation
was reached with just approximately $20-25$ prototypes proving that
the dissimilarity representation can produce compact feature spaces
for this kind of data.

When comparing the different prototype selection policies we found
that FFT had a small advantage over SFF but only when the number of
prototypes was very low ($p<10$). Both FFT and SFF always outperformed
the random policy. Moreover, since the computational cost of SFF does
not increase with the size of the dataset but only with the number of
prototypes, we observed that the SFF policy can be easily computed on
a standard computer even in the case of a tractography of $3\times
10^5$ streamlines. This is different from FFT which is several orders
of magnitude slower than SFF, thus computationally less practical.

We advocate the use of the dissimilarity approximation for the
Euclidean embedding of
tractography data in machine learning and pattern recognition
applications. Moreover we strongly suggest the use of the SFF policy
to obtain an efficient and effective selection of the prototypes.


%%% Local Variables: 
%%% mode: latex
%%% TeX-master: "nguyen_dissimilarity"
%%% End: 
