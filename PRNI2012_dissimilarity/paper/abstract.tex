\begin{abstract}
  Diffusion magnetic resonance imaging (dMRI) data allow to
  reconstruct the 3D pathways of axons within the white matter of the
  brain as a tractography. The analysis of tractographies has drawn
  attention from the machine learning and pattern recognition
  communities providing novel challenges such as finding an
  appropriate representation space for the data. Many of the current
  learning algorithms require the input to be from a vectorial
  space. This requirement contrasts with the intrinsic nature of the
  tractography because its basic elements, called streamlines or
  tracks, have different lengths and different number of points and
  for this reason they cannot be directly represented in a common
  vectorial space. In this work we propose the adoption of the
  dissimilarity representation which is an Euclidean embedding
  technique defined by selecting a set of streamlines called
  prototypes and then mapping any new streamline to the vector of
  distances from prototypes. We investigate the degree of
  approximation of this projection under different prototype selection
  policies and prototype set sizes in order to characterise its use on
  tractography data. Additionally we propose the use of a scalable
  approximation of the most effective prototype selection policy that
  provides fast and accurate dissimilarity approximations of complete
  tractographies.
\end{abstract}

%%% Local Variables: 
%%% mode: latex
%%% TeX-master: "nguyen_dissimilarity"
%%% End: 
