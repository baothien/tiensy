\begin{abstract}
%why is it relevant - important?
%Why is it critic - difficult?
Nowadays, the large multivariate datasets become more and more common. However, traditional multivariate visualization techniques, which although allow to visually analyze and explore data, can not scale well with the large one. This restrain the ability of detecting, recognizing and classifying phenomena of interest, such as patterns, clusters, trends, ect.
%What is our goal?
This paper proposes a general framework for interactive multi-resolution visualization to overcome the problem of traditional visualization techniques when working with a large dataset. The main idea is to develop a multiple abstraction-levels of the data via hierarchical clustering. Based on hierarchical clustering, users can not only examine the dataset at different levels of detail, but also can explore many regions of interest. Beside, the navigation tool and filtering operations create an easy environment for interactive exploration without re-run the clustering algorithm. 
\textbf{not finished yet}
%also   and    , our project, the goal is: first, to design an effective method for tract segmentation task based on \emph{machine learning}
%What is the outline of research?
%We propose a design of the interactive segmentation process, consisting of two steps: tract identification 

 
\end{abstract}

