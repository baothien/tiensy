\begin{abstract}
%why is it relevant - important?
%Why is it critic - difficult?
Nowadays, the large datasets become more and more common. However, traditional visualization techniques, which although allow to visually analyze and explore data, can not scale well with the large one. This restrains the ability of detecting, recognizing and classifying phenomena of interest, such as patterns, clusters, trends, ect.
%What is our goal?
This paper proposes a general framework for interactive multi-resolution visualization to overcome the problem of traditional visualization techniques when working with a large dataset by choosing multiple abstraction-levels for representing data via hierarchical clustering. Based on hierarchical clustering, users can not only examine the dataset at different levels of detail, but also can explore many regions of interest. The basic idea underlying this method is to choose multiple scales from hierarchical tree for representing the data at different levels of abstraction, which creates an easy environment for interactive exploration without re-run the clustering algorithm. We evaluate the proposal method by applying it into the task of \textit{tractography segmentation} in the dMRI data. The results show that our proposed method efficiently provides a friendly and easy tool for visualization the large data.  
\end{abstract}

