\section{Introduction}
\label{sec:introduction}

the need of visualization large data

what is the limitation of current visualization method

- the need of visualization the large dataset $\mathcal{X}$ at different scales, not display all the dataset \textbf{need to investigate}

Contents of the paper
In this work we describe the algorithmic solutions we adopted in order
to build the interactive visualization large data

% tractography segmentation tool. The core of the problem is to obtain fast clustering of a large number of streamlines in order to comply with the requirements of human interaction. The proposed solution combines two state-of-the-art elements: first a recently proposed Euclidean embedding algorithm for streamlines, i.e. the dissimilarity representation with the scalable \emph{subset farthest first} (SFF) prototype selection policy~\cite{olivetti2012approximation}. This embedding provides fast and accurate vectorial representation of streamlines, needed by most of the clustering algorithms. Second, a recently proposed improvement of the $k$-means clustering algorithm called \emph{mini-batch} $k$-means~\cite{sculley2010web} (MBKM). This algorithm drastically reduces the convergence time to the actual clusters in case of large and very-large sets of objects. We also tested a further potential improvement, i.e. a recently proposed initialisation algorithm for the $k$-means family, called $k-means++$~\cite{arthur2007kmeans}. But we excluded it from the final system due to poor performance as illustrated in Section~\ref{sec:experiments}.

% Structure of the paper
The paper is structured as follows. In Section~\ref{sec:methods} the
algorithmic elements of the proposed method are formally described. In
Section~\ref{sec:experiments} we describe details of the actual use of the proposed solution in the
context of the CTS segmentation. 
%We quantitatively describe the segmentation process and provide figures to evaluate the viability of the proposed solution. In Section~\ref{sec:discussion} we discuss the results and we show that the proposed solution is effective. We conclude by mentioning the future steps for the development of our software, called \emph{spaghetti}, and the open computational challenges that needs to be solved.

%%% Local Variables: 
%%% mode: latex
%%% TeX-master: "olivetti_boi"
%%% End: 
