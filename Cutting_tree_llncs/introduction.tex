\section{Introduction}
\label{sec:introduction}
%the need of visualization data
To support human in analyzing and exploring large data, it is an important task to graphically present the data~\cite{yang2003interactive}. Users, in one side, have a requirements of looking at complex and intricate data to find out some facts or trends that are not easy to find. On the other side, they want to explore data in details to examine each data points. In fact, the overall premise is that users have a deeper understanding about their data when they interact with the presented information and view it at different levels of abstraction~\cite{roberts2007state}. During the last two decades, many interactive visualization techniques and system have been emerged~\cite{yang2003interactive,stroe2000scalable,fua2000structure}.
As large data sets become more and more common, with the size over $1K$, it has been clear that most of the current visualization approaches lose their effectiveness due to they have no ability to visualize and manage the large number of data points simultaneously. In such scenario, clustering is considered a suitable method for understanding and exploring large data~\cite{berkhin2006survey,bisson2012improving}. 

However, clustering usually results in one partition of the data, and this leads to a dramatic drawback that the validation process is not straightforward due to the lack of ground truth data~\cite{candillier2006casade}. One solution for this is hierarchical clustering~\cite{johnson1967hierarchical}, which organizes data in an intuitive and interpretable structure, namely \emph{dendrogram}, not only one partition as the traditional clustering methods.
%In such structure, all the degree of generality are present and it 
Such structure allows users to explore in a simple way the clusters and the relationships between instances, and leads to many applications for visualization~\cite{heard2009novel,landesberger2011visual,mahe2009graph}.
%what is the limitation of current visualization method to large data
Nevertheless, when dealing practically with a large size dendrogram, it becomes difficult since the number of nodes grows exponentially with the depth of the tree and makes users lose the overview of the whole dataset. To deal with a large size dendrogram, many approaches have been suggested~\cite{bisson2012improving,landesberger2011visual,furnas2006fisheye}. However, these methods are either display at one time only a sub-part of the structure~\cite{landesberger2011visual,furnas2006fisheye}, or display whole dendrogram but rely on other clustering technique~\cite{bisson2012improving}.

%Contents of the paper
In this work, we propose a method of offering a complete and interactive visualization of the large data based on hierarchical clustering. Our method allow users to apply their perceptual abilities to make sense of data. 
%We describe the algorithmic solutions we adopted in order to build the interactive visualization large data.  
The core of the problem is to obtain the multiple scales representation large data, in order to comply with the requirements of human interactive visualization. The proposed solution combines three steps. First, the dendrogram would be created by running the hierarchical clustering. Second, the \emph{goodness} function is used as a measurement to select the most relevant scales for representing the dendrogram (it is an extension of the "relevant function", proposed in~\cite{pons2011postprocessing}). Lastly, we evaluate the multiple scales based on a 
%application driven 
statistical criteria, called \emph{split factor}. 

\vspace{0.5mm}
Moreover, we conceive an experiment of applying our method in a clinical case study of dMRI data. %dMRI is an imaging modality that captures the diffusion of water in tissues, and along with it, important structural information. %It has been widely used in the study of the connectivity of the human brain connectivity. 
Recently, from dMRI data, tracking algorithms~\cite{mori2002fiber,zhang2008identifying} allow to reconstruct the $3D$ pathways of axons within the white matter % of the brain
as a set of streamlines, called tractography. A \emph{streamline} is a vectorial representation of thousands of neuronal axons expressing structural connectivity, and \emph{tractography} is a set of $N$ streamlines ($N \sim 3 \times 10^5$ usually). It is an important task of segmentation the tractography into some real anatomical structures of interest, such as cortinal spinal tract~\cite{cosottini2010evaluation,sage2009quantitative} involving to the amiotrophic (ALS) disease. In this experiment, we conceive a novel computer-assisted interactive segmentation process based on our method of multiple scales for representation the large tractography.

% Structure of the paper
The paper is organized as follows. Section~\ref{sec:problem} formally introduces the problem of multiple scales for representation large data. After that, Section~\ref{sec:methods} describes the detail of the method for selecting multiple scales. The evaluating the goodness of the representation is presented in the next Section~\ref{sec:evaluation_scales}.
%introduces the algorithmic elements of the proposed method including hierarchical clustering, multiple scales choosing and quantitatively evaluating the goodness of the representation. 
In section~\ref{sec:experiments}, we describe an experiment of applying the proposed solution in the context of the tractgoraphy segmentation, provide figures to evaluate the viability of the proposed solution. We conclude with a summary of our contribution and open areas for future work in the last section~\ref{sec:conclusion}.


