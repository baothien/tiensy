\vspace{-2mm}
\section{Conclusion}
\label{sec:conclusion}
\vspace{-2mm}
%the contribution of the paper
In this paper, we presented a method for addressing the problem faced when attemping to interactively visualize a large dataset. The core principle behinds the framework was to choose \emph{multiple scales for representing} the data from the hierarchical clustering. Moreover, we also proposed a function to evaluate the goodness of each chosen scale based on the concept of \emph{split factor}. We instantiate this framework with an application of building the interactive visualization large dMRI data in the procedure of tractogaphy segmentation, and provide concrete result on it performance. Experiments have shown that our method provide a significant improvement for visualization the large data at different scales, which verifies the effectiveness of the interactive hierarchical visualization. Beside, we are convinced that this method can be easily integrated to any current display techniques without having to vary the data or the interactive exploration tool. 

%what is the limitation of our proposed methods and/or future works
\vspace{1mm}
As mention is the section~\ref{sec:methods}, the level of detail of each cluster, $s(C_i)$, can be computed based on radius or heigh~\cite{yang2003interactive}. In this paper we choose the multiple scales only based on the heigh of cluster. The same job but based on the radius needs to be investigated. Moreover, this work is a part of an going resarch project focusing on computer-aided tractography segmentation, where machine learning techniques are used to assist medical practitioners to do the segmentaion task more easily, flexibly  and effectively. In the future, we want to further improve the interactive segmentation tool by providing the function of adding or eleminating data points $x$ into or from the current dataset $\mathcal{X}$, and updating the visualization result without re-run the clustering algorithm.

\vspace{-3mm}

