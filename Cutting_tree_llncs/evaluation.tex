\vspace{-2.5mm}
\section{Evaluation}
\label{sec:evaluation_scales}
\vspace{-1.85mm}
In this part we describe criteria for evaluating the multiple scales representation $\mathsf{\textit{B}} = \{b_1, b_2, \ldots, b_k\}$ for the dataset $\mathcal{X}$. It is the real fact that, at a certain time, users can only exam about the total of $~50$ $(\lambda_1)$ clusters which are currently displaying on the screen. Among of the visible clusters, users usually select around $~15$ $(\lambda_2)$ clusters to explore or exam the data~\cite{olivetti2013fast}. Driven from that, we propose a method to evaluate the represented multi-scale set $\mathsf{\textit{B}}$ based on the \textit{split factor} as following. 
%using our software tool to do the segmentation of cortinal spinal tract, medical practitioners usually choose $~15$ $(\lambda_1)$ clusters from $~50$ $(\lambda_2)$ representatives~\cite{olivetti2013fast}, we propose a method to evaluate the represented multi-scale set $\mathsf{\textit{B}}$ based on the \textit{split factor} as following.
\begin{definition}
\label{def:split_factor} \textbf{\textit{Split factor $\xi$}} of \textit{\textbf{a cluster}} $C \in \mathcal{H}$ to a scale $w \in [s_{min},s_{max}]$ is $\xi(C,s)$
\begin{equation}
\xi(C,s) = card(P(C,s))
\end{equation}
where $P(C,s) = \{ C_j | (C_j \in \mathcal{H}) \wedge (s(C_j) = w) \wedge (C_j \subseteq C) \}$ 
\end{definition}
\begin{definition}
\label{def:split_factor_set} \textbf{\textit{Split factor $\xi$}} of a \textit{\textbf{set of clusters}} $P=\{C_1, C_2,.., C_m\} \subseteq \mathcal{H}$ to a scale $s \in [s_{min},s_{max}]$ is $\xi(P,s)$
\begin{equation}
\xi(P,s) = \sum_{C_i \in P}\xi(C_i,s)
\end{equation}
\end{definition}
%Given a specific scale $w \in \mathsf{\textit{B}}$, draw uniformly at random $k_1$ clusters from the cut $\mathfrak{L}(w)$ of tree $\mathcal{H}$ at $w$, called $S_w$: $S_{w}=\{C_{s_1}, C_{s_2},.., C_{s_{k_1}}\}$, $C_{s_{i=1}^{k_1}} \in \mathfrak{L}(w)$.
\begin{definition}
\label{def:best_scales} The set of scales $\mathsf{\textit{B}} = \{b_1, b_2, \ldots, b_k\}$ is called \textbf{\textit{the best scales for representation}} of the tree $\mathcal{H}$, given $\lambda_1$ and $\lambda_2$, if the following condition satisfies
\begin{equation}
\label{equ:best_scale}
\forall b_i \in \mathsf{\textit{B}}: \lambda_1 - \Delta \leq \xi(S_{(b_i,\lambda_2)},b_{i-1}) \leq \lambda_1 + \Delta 
\end{equation}
where $S_{(b_i,\lambda_2)}$ is a Gaussian distribution subset of the cut $\mathcal{H}$ at scale $b_i$, $\mathfrak{L(b_i)}$, with the order of $\lambda_2$:
\begin{equation}
S_{(b_i,\lambda_2)} = \{C_1, \ldots, C_{\lambda_2}\}, C_j \in \mathfrak{L(b_i)}, \forall j \in [1, \ldots, \lambda_2] 
\end{equation}
\end{definition}
In the case of $b_1$, the split factor is computed to the leaf: $\xi(S_{(b_1,\lambda_2)},0)$. The pseudo code of evaluation procedure is presented in algorithm~\ref{alg:evaluate}
%the best cut scale set \mathsf{B}% = {b_1, b_2, \ldots, b_k}}
%\mathcal{H}
%\IncMargin{1em}
%\begin{algorithm}
	%\SetAlgoLined
%	\AlgoDisplayBlockMarkers\SetAlgoNoLine%
%	\SetKwData{Left}{left}\SetKwData{This}{this}\SetKwData{Up}{up}
%	\SetKwFunction{Union}{Union}\SetKwFunction{FindCompress}{FindCompress}
%	\SetKwInOut{Input}{input}\SetKwInOut{Output}{output}
%	\Input{a hierarchical tree $\mathcal{H}$\\
%	a set of cut scales $\textit{$\mathsf{B}$} = \{b_1, b_2, \ldots, b_k\}$}
%	\Output{accept \textit{$\mathsf{B}$} as a good represent for $\mathcal{H}$ or not}
%	\BlankLine
%	\tcc{initialization}
%	$accept \leftarrow True$\;
%	$i \leftarrow 1$\;
%	\While {$(i \leq k-1) \wedge (accept)$ }
%	{
%		$w \leftarrow b_i$\;
%
%		$l \leftarrow b_{i+1}$\;
%
%		%\tcc{draw $S=\{s_1,s_2,..,s_{k_1}\}$ of size $k_1$ uniformly at random}
%
%		$ S \leftarrow$ choose $\lambda_1$ clusters uniformly at random from $\mathfrak{L}$$(w)$\;
%
%		\tcc{calculate the split factor of $S$ to scale $l$}
%
%		$t \leftarrow \xi(S,l)$\;
%
%		\tcc{check the quality of the cut $b_i$ based on equation~\ref{equ:best_scale}}
%		\If{$(t \geq \lambda_2 + \Delta ) \vee (t \leq \lambda_2 - \Delta) $}
%		{
%			\tcc{update the result}
%			$accept \leftarrow False$\;
%		}		
%		$i \leftarrow i+1$\;
%	}
%	return $accept$\;
%\caption{Evaluate the set of cut scales, based on split factor}\label{algo_evaluate}
%\end{algorithm}
%\DecMargin{1em}
\vspace{-4mm}
\algsetup{indent=2.75em}
\begin{algorithm}[h!]
\caption{Evaluate the set of cut scales, based on split factor}\label{alg:evaluate}
\begin{algorithmic}[1]
	\REQUIRE a hierarchical tree $\mathcal{H}$ \AND \\
             \hspace{8 mm} a set of cut scales $\textit{$\mathsf{B}$} = \{b_1, b_2, \ldots, b_k\}$
	\ENSURE accept \textit{$\mathsf{B}$} as a good represent for $\mathcal{H}$ or not
	\medskip
	\STATE $accept \leftarrow \TRUE$ \COMMENT{initialization}
	\STATE $i \leftarrow 1$
	\WHILE {$(i \leq k-1)$ \AND $(accept)$ }
		\vspace{0.8mm}
		\STATE $w \leftarrow b_i$

		\STATE $l \leftarrow b_{i+1}$
		\vspace{1.mm}
		\STATE $ S \leftarrow$ choose $\lambda_2$ clusters uniformly at random \\
								\hspace{8mm} from $\mathfrak{L}$$(w)$
		\vspace{1.mm}
		\STATE $t \leftarrow \xi(S,l)$ \COMMENT{split factor of $S$ to scale $l$}\\
		\vspace{1.mm}
		\COMMENT{check quality of the cut $b_i$ based on equation~\ref{equ:best_scale}}
		\IF{$(t \geq \lambda_1 + \Delta )$ \AND $(t \leq \lambda_1 - \Delta) $}
			\vspace{0.8mm}	
			\STATE $accept \leftarrow \FALSE$ \COMMENT{update the result}
		\ENDIF
		\vspace{0.9mm}
		\STATE $i \leftarrow i+1$\;	
	\ENDWHILE
	\vspace{0.9mm}
	\RETURN $accept$
\end{algorithmic}
\end{algorithm}
