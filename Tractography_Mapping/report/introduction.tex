\section{Introduction}
\label{sec:introduction}

In this work we explore the idea of \textbf{mapping} each streamline
($s^S_i$, $i \in 1 \ldots M$) of one \emph{source} tractography ($T_S
= \{s^S_1,\ldots,s^S_M\}$) into one streamline of another
\emph{target} tractography ($T_T = \{s^T_1,\ldots,s^T_N\}$). The
mapping $\phi: T_S \mapsto T_T$ just associates the first object to
the second, e.g. $s^2_j = \phi(s^1_i)$, or more simply $j = \phi(i)$.

The practical application that motivates the idea of mapping is
related to tractography coregistration. While tractography
coregistration is based on rigid or non-rigid shape transformations of
one tractography into another, with the goal of making them the most
similar, in \emph{mapping} the idea is that no transformation is
meaningful because it is not possible to reconcile two different
anatomies by means of rigid (or non-rigid) transformations. The idea
of mapping is to find to which target streamline a source streamline
corresponds. This may be done with rigid or non-rigid transformations,
but here we are interested in a different and indirect approach that
does not require such transformations.

\subsection{Loss Function}
\begin{equation}
  \label{eq:loss}
  l = \sum_{i \neq j \in 1 \ldots M} (d(s^S_i, s^S_j) -
  d(s^T_{\phi(i)}, s^T_{\phi(j)}))^2
\end{equation}



%%% Local Variables: 
%%% mode: latex
%%% TeX-master: "tractography_mapping"
%%% End: 
