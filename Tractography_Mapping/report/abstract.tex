Abstract

Diffusion tensor imaging (DTI) and tractography have opened up new avenues in neuroscience. As most applications require precise spatial localization of the fibre images, image registration is an important area of research. Registration of tractography is most often performed by applying the
transformation resulting from an image-based fractional anisotropy (FA) or diffusion
tensor imaging (DTI) registration method \cite{goodlett2009group, dan2011comparision, wang2011dti}. Recently, O’Donnell et al. proposed an
unbiased multiple subject registration method using the trajectory data produced by
streamline tractography~\cite{odonnell2012unbiased}. The idea to work on deterministic tractography rather than
dMRI images or FA images is quite innovative, and it may be advantageous to register
the tracts themselves as the quantity being optimized would be closely related to the final
goal. However, in this work we explore the idea of \textbf{mapping} each streamline
($s^S_i$, $i \in 1 \ldots M$) of one \emph{source} tractography ($T_S
= \{s^S_1,\ldots,s^S_M\}$) into one streamline of another
\emph{target} tractography ($T_T = \{s^T_1,\ldots,s^T_N\}$) dirrectly. The
mapping $\phi: T_S \mapsto T_T$ just associates the first object to
the second, e.g. $s^2_j = \phi(s^1_i)$, or more simply $j = \phi(i)$. While tractography
coregistration is based on rigid or non-rigid shape transformations of
one tractography into another, with the goal of making them the most
similar, in \emph{mapping} the idea is that no transformation is
meaningful because it is not possible to reconcile two different
anatomies by means of rigid (or non-rigid) transformations. The idea
of mapping is to find to which target streamline a source streamline
corresponds. This may be done with rigid or non-rigid transformations,
but here we are interested in a different and indirect approach that
does not require such transformations.



%%% Local Variables: 
%%% mode: latex
%%% TeX-master: "tractography_mapping"
%%% End: 
