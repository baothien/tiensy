\section{Conclusion and Future works}
\label{sec:conclusion}
In this document we investigate the using of machine learning %techniques in neuroimaging 
for tractogarphy segmentation task.
% (both supervised and unsupervised) for tractogarphy segmentation task.
We design the framework of interactive segmentation based on two steps: tract identification and tract refinement. 
%We propose a design of interactive segmentation process based on two steps: tract identification and tract refinement. 
%we design an effective method for tract segmentation task based on \emph{machine learning}. We propose a framework for this task including two main steps: hypo generation and tract refinement. 
The first step produces the the initial candidate 
%of segmentation 
from a full tractography by supervised learning. The second one aims to refine the candidate by removing selected streamlines or adding more 
%including additional 
ones, and is conceived as a clustering task. Our main focus is on this step rather than the previous one, which can be done by using the method proposed in~\cite{olivetti2011supervised}. 

First, we propose a solution for clustering tractography based on Hierarchical clustering, which meets many requirements in our case. Second, we study the dissimilarity approximation for tractography, to present streamlines in a vectorial space.
%, which is required for all current clustering algorithm. 
%This representation works well for preserving the relative distances and be able to produce compact feature spaces for tractography data.
We also provide an implementation of our proposed two-step framework for tract segmentation, a streamline interaction tool, called Spaghetti. In this scientific interaction tool, we present an simple way to interact and segment streamlines in $3D$ space, which goes, as far as we know, beyond any other available medical imaging software.
% Moreover, this demonstration also integrates many utility functions, such as undo, log, zoom, save the works, load the result, etc. 
This enables medical practitioners and researchers to meaningfully navigate the entire space of the tractography and perform the segmentation task more easily and accuracy.

However, up to present, we still represent streamlines in the original space (a polyline $s=\{\vec{x_1},\ldots,\vec{x}_{n_s}\}$, where $\vec{x} \in \mathbb{R}^3$).
%, and to calculate the distance between streamline and streamline, we use mam~\ref{eq:mam_distance} or mdf~\ref{eq:mdf_distance}. 
This leads to a fact that the streamline distance(\textit{mam}~\ref{eq:mam_distance} or \textit{mdf}~\ref{eq:mdf_distance}) sometimes is not a \textit{real distance}. For example, the distance between a very short streamline ($n_s$ small) and a long streamline ($n_s$ large) is really large, no matter what two streamlines are far each other or not. This drawback should be dismissed when we represent streamline in the dissimilarity approximation. Converting from original space into dissimilarity approximation space, and define a new distance measurement is one of our future works. Beside, in the current version of Spaghetti, we use QB to cluster streamlines. Although we propose HAC clustering and point out many advantages of it in our case, but we have not integrated it into Spaghetti. In a near future, QB should be replaced by HAC. Moreover, Hierarchical creates a tree of nested cluster, it also needs to design a suitable structure for storing and accessing these clusters as well. Beside, we have not found the solution for the \emph{neighbor checking} problem yet. It is also another future work.
%from  dMRI pre-processing, dissimilarity approximation presentation and applying learning techniques - \textbf{fast clustering} to segment tractography. The result of tractography segmentation task can be used for clinical diagnosis applications. Because dMRI pre-processing is the compulsory step for almost study in neurodata analyses\cite{garyfallidis2012towards}, in this paper we present no result of this stage. More detail about this can be found in the technical report of ~\cite{bao2012dmri}. 
 %In this work, we investigate the degree of approximation of the dissimilarity representation for the goal of preserving the relative distances between streamlines within tractographies. Empirical assessment has been conducted on two different datasets and through various prototype selection methods. The results from real tractography data reached correlation $\ge 0.95$ with respect to the distances in the original space. This fact proved that the dissimilarity representation works well for preserving the relative distances. Moreover on tractography data the maximum correlation was reached with just approximately $20-25$ prototypes proving that the dissimilarity representation can produce compact feature spaces for this kind of data.
%some text about sphagetti here
%We presented a novel visualization tool which provides an entirely original way to interact with very large tractogaphies. This goes far beyond the methods used in [4-8] and, as far as we know, beyond any other available medical imaging software. We provide a software implementation of this application which is written completely in Python language we show results of interactive tractography exploration using both BOIs and ROIs with user interactivity directly in the 3D space. This enables medical practitioners and researchers to meaningfully navigate the entire space of the tractography and to better understand diffusion MRI data. The fast clustering algorithm is used in an interactive viusalization tool. In this scientific visualization tool, we solves the problem of interacting with tractographies by creating real-time simplifications in terms of the underlying bundle structures. The process that we propose works recursively: starting from a small number of clusters of streamlines the user decides which clusters to explore. Exploring a cluster means that the application re-clusters its content at a finer grained level
%~\cite{garyfallidis2012software}.
%To calculate the distance between to streamlines, in this tool, we have not applied the dissimilarity approximation yet. We use the two common streamline distances: 
About the clinical diagnosis application, we believe that the satisfactory result will be published in near future. After that, the general framework for clinical diagnosing based on the differences between two folders of interesting tracks must be presented and applied for other brain diseases.

%This results provide empirical evidence that the dissimilarity space representation is effective for the fiber tract segmentation. In future, we will work step on the using of dissimilarity approximation for tractography segmentation based on machine learning techniques. Moreover, we want to propose a general algorithm for the tractography segmentation problem which can get the hight accuracy and less computational cost.

