\section{Introduction}
\label{sec:introduction}

% What is dMRI and tractography
Diffusion magnetic resonance imaging
(dMRI)~\cite{basser1994mr} techniques provide
non-invasive images of the brain white matter. dMRI quantifies
locally, i.e. in each voxel, the diffusion process of the water
molecules which are mechanically constrained in their motion by the
axons of the neurons. Reconstruction and tracking
algorithms~\cite{mori2002fiber} transform dMRI data into a set of
streamlines, i.e. 3D polylines that approximate the neuronal
pathways. The whole set of streamlines is called tractography and it
represents the anatomical connectivity of the brain. See for example
Figure~\ref{fig}.

% % General clinical motivation for using dMRI
% REMOVE THIS PARAGRAPH. The investigation of white-matter anatomical
% structures plays a central role in understanding and characterising
% diseases such as multiple sclerosis and Alzheimer's disease. The
% hypothesis of a reduction in the anatomical connectivity is considered
% as one of the possible causes of them. The quantification of this
% reduction may be done from the analysis of tractography data and for
% this reason the segmentation of the tractography into known
% anatomically tract is a task of interest in neurological
% studies~\cite{clayden2009reproducibility}.

% Our specific motivation: tractography segmentation to study ALS
This work focuses on computer-assisted tractography segmentation and
describes the solution we developed to build a software system to
support neuroanatomists and medical doctors in studying the white
matter. Our work is motivated by a clinical research hypothesis about
the characterisation of the amiotrophic lateral sclerosis (ALS)
disease. We collected dMRI data from ALS patients and healthy controls
with the aim of studying the effects of the ALS disease on the
corticospinal tract (CST) (see Figure~\ref{fig}I), an anatomical
structure that connects cortical motor areas to the spine and the
body. The CST is known to be affected by
ALS~\cite{cosottini2010evaluation} and for this
reason, our long term goal is to characterise these effects through
tractography data. The first task in this endeavour is to segment the
CST from the full brain tractography of each subject.

% Brief description of the proposed computer-assisted segmentation process
Tractography segmentation can be performed manually or
automatically. Despite an increasing literature in automatic
segmentation (see a brief review
in~\cite{wang2011tractography}), the
application in the clinical domain usually rely on manual
segmentation. The manual segmentation process usually consists in
selecting the subset of the streamlines connecting a few manually
located regions of interest\footnote{See for example \url{http://www.trackvis.org}.}. This task is a lengthy and complex one,
for two reasons: first the tractography is a very large set of
streamlines, in the order of $3 \times 10^5$, which makes it
intrinsically difficult both to inspect and to unfold anatomical
structures (see Figure~\ref{fig}A). Second, the reconstruction of the
streamlines is frequently suboptimal due to the noise in the
measurement process and to the limitations of reconstruction
algorithms. For this reason, a single neuronal pathway may be just
partially reconstructed, resulting in multiple disconnected
polylines. These \emph{broken} streamlines would be discarded by the
manual segmentation procedure mentioned above, because not
connecting the regions of interest defined by the expert.

Differently from the previous approaches of tractography segmentation,
we conceived a novel computer-assisted interactive process based on
clustering algorithms, which aims at greatly reducing the time
required to manually segment a given anatomical white matter structure
of interest. 
% SHORT VERSION (SEE LONG VERSION BELOW):
Our approach is based on a fast-clustering technique by means of which
the expert is presented with a summary of the streamlines, i.e. the
clusters represented by their medoids~\footnote{A medoid is the
  element of a cluster closest to its centre.}. The expert manually
selects the medoids/clusters of interest in order to remove most of
the streamlines not related to the anatomical structure of interest.
Interacting with the summary, instead of the actual streamlines, is
much simpler for the user. In the example of Figure~\ref{fig}, the
user selects $15000$ streamlines (see \ref{fig}C) of the $3 \times
10^5$ streamlines (see \ref{fig}A) just by clicking on $20$ of the
$150$ medoids (see \ref{fig}B, the selected medoids are shown in
white).  The process of reclustering the selected streamlines and of
manual selection by the expert is iterated until the expert is
confident of having segmented the structure of interest (see
Figure~\ref{fig}I).
% THIS IS THE LONG VERSION OF THE PREVIOUS PARAGRAPH:
% Our approach is based on a fast-clustering technique by means of which
% the expert is initially presented a summary, i.e. the
% medoids~\footnote{A medoid is the element of a cluster closest to the
%   centroid.} of the clusters (see for example Figure~\ref{fig}B), of
% the whole tractography where she can manually select the subset of
% clusters to which to restrict the attention to. The selected subset of
% medoids (see them in white in Figure~\ref{fig}B) is an initial rough
% approximation the streamlines of interest (depicted in
% Figure~\ref{fig}C). This set is then re-clustered with an increased
% number of clusters so that more anatomical details are revealed (see
% Figure~\ref{fig}D). There the user can select what cluster is of
% interest and discard what is not. The selected clusters are then
% re-clustered in more and smaller clusters to reveal more details of
% the anatomical structure of interest. The selection and re-clustering
% process is repeated iteratively until the expert is confident of
% having segmented the structure of interest (see for example
% Figure~\ref{fig}I).

% Contents of the paper
In this work we describe the algorithmic solution we adopted in order
to build the interactive tractography segmentation tool. The core of
the problem is to cluster a large number of streamlines in no more
than a few seconds, to allow a comfortable interactive user experience
to the expert. The proposed solution combines two state-of-the-art
elements: first a recently proposed Euclidean embedding algorithm for
streamlines, i.e. the dissimilarity representation with the scalable
\emph{subset farthest first} (SFF) prototype selection
policy~\cite{olivetti2012approximation}. This embedding provides fast
and accurate vectorial representation of streamlines. Second, a
recently proposed improvement of the $k$-means clustering algorithm
called \emph{mini-batch} $k$-means~\cite{sculley2010web} (MBKM). This
algorithm, which require the data to lie in a vector space,
drastically reduces the convergence time to the actual clusters in
case of large and very-large sets of objects. We claim that the
dissimilarity embedding together with the MBKM algorithm provides a
viable solution to problem of fast clustering of streamlines.
% REMOVE
% We also tested a further potential
% improvement, i.e. a recently proposed initialisation algorithm for the
% $k$-means family, called $k$-means++~\cite{arthur2007kmeans}. But we
% excluded it from the final system due to poor performance as
% illustrated in Section~\ref{sec:experiments}.

% Structure of the paper
The paper is structured as follows. In Section~\ref{sec:methods} the
algorithmic elements of the proposed method are formally described. In
Section~\ref{sec:experiments} we describe the segmentation process and
report the details of the actual use of the proposed solution in the
context of the CST segmentation. We quantitatively describe the
segmentation process and provide timings to evaluate the viability of
the proposed solution. In Section~\ref{sec:discussion} we discuss the
results and we show that the proposed solution confirms our claims. 
% We conclude by mentioning the future steps for the development of our
% software and the open computational challenges that needs to be
% solved.

%%% Local Variables: 
%%% mode: latex
%%% TeX-master: "olivetti_boi"
%%% End: 
