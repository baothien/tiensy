\section{Introduction}
\label{sec:introduction}

Diffusion MRI (dMRI) techniques produces non-invasive images of the
brain white matter. dMRI quantifies locally, i.e. in each voxel, the
diffusion process of the water molecules which is mechanically
constrained by the axons of the cortical neurons. Reconstruction and
tracking algorithms...

From~\cite{clayden2009reproducibility} REPHRASE: ``A rapidly
accumulating clinical literature based on the technique of diffusion
magnetic resonance imaging (dMRI; see Le Bihan, 2003) is lending
weight to the proposition that the brain’s white matter fasciculi may
be be detrimentally affected in a broad spectrum of pathological
scenarios. The development of diffusion tensor imaging (Basser et al.,
1994), and of derived measures such as fractional anisotropy (Basser
\& Pierpaoli, 1996), has provided tools for gaining insight into the
microstructural properties of white matter in vivo. Clinical
applications for these tools include a range of white matter diseases
such as multiple sclerosis and Alzheimer’s disease (Horsfield \&
Jones, 2002), as well as psychiatric disorders like schizophrenia and
depression (Lim \& Helpern, 2002). Some of these are thought to be
caused, at least partly, by a breakdown in the connective efficacy of
white matter. Such pathologies are known as disconnection syndromes, a
denomination due to Geschwind (1965a,b).''

This work focuses on computer-assisted tractography segmentation and
describes a software system we developed to support neuroanatomists
and medical doctors in studying the white matter.

Tractography segmentation can be performed manually,
semi-automatically and automatically. Some comments and references (if
available) on each modality.

Differently from the previous approaches in manual and assisted
segmentation we present the user a summary of the tractography in
terms of clustering of the whole set of streamlines. DESCRIPTION OF
THE PROCESS.

In the following we describe STRUCTURE OF THE PAPER.

%%% Local Variables: 
%%% mode: latex
%%% TeX-master: "olivetti_boi"
%%% End: 
