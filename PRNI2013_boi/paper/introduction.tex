\section{Introduction}
\label{sec:introduction}

% What is dMRI and tractography
Diffusion magnetic resonance imaging (dMRI) (REFERENCE) techniques
provide non-invasive images of the brain white matter. dMRI quantifies
locally, i.e. in each voxel, the diffusion process of the water
molecules which is mechanically constrained by the axons of the
cortical neurons. Reconstruction and tracking algorithms (REFERENCE)
transform dMRI data into a set of streamlines, where a streamline is a
3D reconstruction of neuronal pathways. The whole set of streamlines
is called deterministic tractography. See for example
Figure~\ref{fig:}

% General clinical motivation for using dMRI
From~\cite{clayden2009reproducibility} REPHRASE: ``A rapidly
accumulating clinical literature based on the technique of diffusion
magnetic resonance imaging (dMRI; see Le Bihan, 2003) is lending
weight to the proposition that the brain’s white matter fasciculi may
be be detrimentally affected in a broad spectrum of pathological
scenarios. The development of diffusion tensor imaging (Basser et al.,
1994), and of derived measures such as fractional anisotropy (Basser
\& Pierpaoli, 1996), has provided tools for gaining insight into the
microstructural properties of white matter in vivo. Clinical
applications for these tools include a range of white matter diseases
such as multiple sclerosis and Alzheimer’s disease (Horsfield \&
Jones, 2002), as well as psychiatric disorders like schizophrenia and
depression (Lim \& Helpern, 2002). Some of these are thought to be
caused, at least partly, by a breakdown in the connective efficacy of
white matter. Such pathologies are known as disconnection syndromes, a
denomination due to Geschwind (1965a,b).''

% Our specific motivation: tractography segmentation to study ALS
This work focuses on computer-assisted tractography segmentation and
describes the solutions we developed to build a software system we
created to support neuroanatomists and medical doctors in studying the
white matter. Our work is motivated by a clinical research hypothesis
about the characterisation of the amiotrophic (ALS) disease. We
collected dMRI data from ALS patients and healthy controls with the
aim of studying the effects of the ALS disease on the corticospinal
tract (CST), which connects cortical motor areas to the spine and the
body. The CST is known to be affected by ALS (REFERENCE) and for this
reason our long term goal is to characterise these effects in
tractography data. The first task in this endeavour is to segment the
CTS from the full brain tractography for each subject.

% Brief description of the proposed computer-assisted segmentation process
Tractography segmentation can be performed manually or
automatically. Despite an increasing literature in automatic
segmentation (see a brief review in~\cite{WANG?}), applications in the
clinical domain rely on manual segmentation. The manual segmentation
process is a lengthy and complex task for two reasons: first the
tractography is a very large set of reconstructed neuronal pathways,
in the order of $3 \times 10^5$, which make it intrinsically difficult
both to inspect and to unfold the anatomical structures (see
Figure~\ref{fig:}. Second, we claim that there is a lack of software
tools to support and to simply this segmentation process.

Differently from the previous approaches to tractography segmentation
we conceived a novel computer-assisted interactive process which aims
at greatly reducing the time required by an expert to segment a given
anatomical white matter structure of interest. Our approach is based
on fast-clustering techniques by means of which the expert is
initially presented a summary, i.e. a clustering, of the whole
tractography where she can easily select the subset of clusters to
which restrict the attention. The selected subset of streamlines is an
initial rough approximation of the structure of interest. This set is
then re-clustered with an increased number of clusters so that further
details are revealed. There the user can select what cluster is of
interest and discard what is not. The selected clusters are then
re-clustered in more and smaller clusters to reveal more details of
the anatomical structure of interest. The selection and re-clustering
process is repeated iteratively until the expert is confident of
having segmented the structure of interest. A practical example of
this process is presented in Figure~\ref{fig:}


% Contents of the paper
In this work we describe the algorithmic solutions we adopted in order
to build the interactive tractography segmentation tool. The core of
the problem is to obtain fast clustering of a large number of
streamlines in order to comply with the requirements of human
interaction. The proposed solution combines two state-of-the-art
elements: first a recently proposed Euclidean embedding algorithm for
streamlines, i.e. the dissimilarity representation with the scalable
\emph{subset farthest first} (SFF) prototype selection
policy~\cite{olivetti2012dissimilarity}. This embedding provides fast
and accurate vectorial representation of streamlines, needed by most
of the clustering algorithms. Second, a recently proposed improvement
of the $k$-means clustering algorithm called \emph{mini-batch}
$k$-means~\cite{sculley2010web} (MBKM). This algorithm drastically
reduces the convergence time to the actual clusters in case of large
and very-large sets of objects. We also tested a further potential
improvement, i.e. a recently(?) proposed initialisation algorithm for
the $k$-means family, called $k-means++$~\cite{arthur2007kmeans}. But
we excluded it from the final system due to poor performance as
illustrated in Section~\ref{sec:experiments}.

% Structure of the paper
The paper is structured as follows. In Section~\ref{sec:methods} the
algorithmic elements of the proposed method are formally described. In
Section~\ref{sec:experiments} we describe the segmentation process and
report the details of the actual use of the proposed solution in the
context of the CTS segmentation. We quantitatively describe the
segmentation process and provide figures to evaluate the viability of
the proposed solution. In Section~\ref{sec:discussion} we discuss the
results and we show that the proposed solution is effective. We
conclude by mentioning the future steps for the development of
\emph{spaghetti} and the open computational (pattern recognition?)
challenges that needs to be solved.

%%% Local Variables: 
%%% mode: latex
%%% TeX-master: "olivetti_boi"
%%% End: 
