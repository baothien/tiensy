\documentclass[10pt, conference, compsocconf]{IEEEtran}
% Add the compsocconf option for Computer Society conferences.

% *** MISC UTILITY PACKAGES ***
%
%\usepackage{ifpdf}
% Heiko Oberdiek's ifpdf.sty is very useful if you need conditional
% compilation based on whether the output is pdf or dvi.
% usage:
% \ifpdf
%   % pdf code
% \else
%   % dvi code
% \fi
% The latest version of ifpdf.sty can be obtained from:
% http://www.ctan.org/tex-archive/macros/latex/contrib/oberdiek/
% Also, note that IEEEtran.cls V1.7 and later provides a builtin
% \ifCLASSINFOpdf conditional that works the same way.
% When switching from latex to pdflatex and vice-versa, the compiler may
% have to be run twice to clear warning/error messages.

% *** CITATION PACKAGES ***
%
\usepackage{cite}
% cite.sty was written by Donald Arseneau
% V1.6 and later of IEEEtran pre-defines the format of the cite.sty package
% \cite{} output to follow that of IEEE. Loading the cite package will
% result in citation numbers being automatically sorted and properly
% "compressed/ranged". e.g., [1], [9], [2], [7], [5], [6] without using
% cite.sty will become [1], [2], [5]--[7], [9] using cite.sty. cite.sty's
% \cite will automatically add leading space, if needed. Use cite.sty's
% noadjust option (cite.sty V3.8 and later) if you want to turn this off.
% cite.sty is already installed on most LaTeX systems. Be sure and use
% version 4.0 (2003-05-27) and later if using hyperref.sty. cite.sty does
% not currently provide for hyperlinked citations.
% The latest version can be obtained at:
% http://www.ctan.org/tex-archive/macros/latex/contrib/cite/
% The documentation is contained in the cite.sty file itself.

% *** GRAPHICS RELATED PACKAGES ***
%
\usepackage{subfigure}
\ifCLASSINFOpdf
  \usepackage[pdftex]{graphicx}
  % declare the path(s) where your graphic files are
  % \graphicspath{{../pdf/}{../jpeg/}}
  % and their extensions so you won't have to specify these with
  % every instance of \includegraphics
  % \DeclareGraphicsExtensions{.pdf,.jpeg,.png}
\else
  % or other class option (dvipsone, dvipdf, if not using dvips). graphicx
  % will default to the driver specified in the system graphics.cfg if no
  % driver is specified.
  % \usepackage[dvips]{graphicx}
  % declare the path(s) where your graphic files are
  % \graphicspath{{../eps/}}
  % and their extensions so you won't have to specify these with
  % every instance of \includegraphics
  % \DeclareGraphicsExtensions{.eps}
\fi

% *** MATH PACKAGES ***
%
\usepackage[cmex10]{amsmath}
\usepackage{amsfonts}   % Added by P.Avesani
% A popular package from the American Mathematical Society that provides
% many useful and powerful commands for dealing with mathematics. If using
% it, be sure to load this package with the cmex10 option to ensure that
% only type 1 fonts will utilized at all point sizes. Without this option,
% it is possible that some math symbols, particularly those within
% footnotes, will be rendered in bitmap form which will result in a
% document that can not be IEEE Xplore compliant!

% *** SPECIALIZED LIST PACKAGES ***
%
\usepackage[noend]{algorithmic}
% algorithmic.sty was written by Peter Williams and Rogerio Brito.
% This package provides an algorithmic environment fo describing algorithms.
% You can use the algorithmic environment in-text or within a figure
% environment to provide for a floating algorithm. Do NOT use the algorithm
% floating environment provided by algorithm.sty (by the same authors) or
% algorithm2e.sty (by Christophe Fiorio) as IEEE does not use dedicated
% algorithm float types and packages that provide these will not provide
% correct IEEE style captions. The latest version and documentation of
% algorithmic.sty can be obtained at:
% http://www.ctan.org/tex-archive/macros/latex/contrib/algorithms/
% There is also a support site at:
% http://algorithms.berlios.de/index.html
% Also of interest may be the (relatively newer and more customizable)
% algorithmicx.sty package by Szasz Janos:
% http://www.ctan.org/tex-archive/macros/latex/contrib/algorithmicx/




% *** ALIGNMENT PACKAGES ***
%
\usepackage{array}
\usepackage{multirow}
% Frank Mittelbach's and David Carlisle's array.sty patches and improves
% the standard LaTeX2e array and tabular environments to provide better
% appearance and additional user controls. As the default LaTeX2e table
% generation code is lacking to the point of almost being broken with
% respect to the quality of the end results, all users are strongly
% advised to use an enhanced (at the very least that provided by array.sty)
% set of table tools. array.sty is already installed on most systems. The
% latest version and documentation can be obtained at:
% http://www.ctan.org/tex-archive/macros/latex/required/tools/


%\usepackage{mdwmath}
%\usepackage{mdwtab}
% Also highly recommended is Mark Wooding's extremely powerful MDW tools,
% especially mdwmath.sty and mdwtab.sty which are used to format equations
% and tables, respectively. The MDWtools set is already installed on most
% LaTeX systems. The lastest version and documentation is available at:
% http://www.ctan.org/tex-archive/macros/latex/contrib/mdwtools/


% IEEEtran contains the IEEEeqnarray family of commands that can be used to
% generate multiline equations as well as matrices, tables, etc., of high
% quality.


%\usepackage{eqparbox}
% Also of notable interest is Scott Pakin's eqparbox package for creating
% (automatically sized) equal width boxes - aka "natural width parboxes".
% Available at:
% http://www.ctan.org/tex-archive/macros/latex/contrib/eqparbox/





% *** SUBFIGURE PACKAGES ***
%\usepackage[tight,footnotesize]{subfigure}
% subfigure.sty was written by Steven Douglas Cochran. This package makes it
% easy to put subfigures in your figures. e.g., "Figure 1a and 1b". For IEEE
% work, it is a good idea to load it with the tight package option to reduce
% the amount of white space around the subfigures. subfigure.sty is already
% installed on most LaTeX systems. The latest version and documentation can
% be obtained at:
% http://www.ctan.org/tex-archive/obsolete/macros/latex/contrib/subfigure/
% subfigure.sty has been superceeded by subfig.sty.



%\usepackage[caption=false]{caption}
%\usepackage[font=footnotesize]{subfig}
% subfig.sty, also written by Steven Douglas Cochran, is the modern
% replacement for subfigure.sty. However, subfig.sty requires and
% automatically loads Axel Sommerfeldt's caption.sty which will override
% IEEEtran.cls handling of captions and this will result in nonIEEE style
% figure/table captions. To prevent this problem, be sure and preload
% caption.sty with its "caption=false" package option. This is will preserve
% IEEEtran.cls handing of captions. Version 1.3 (2005/06/28) and later 
% (recommended due to many improvements over 1.2) of subfig.sty supports
% the caption=false option directly:
%\usepackage[caption=false,font=footnotesize]{subfig}
%
% The latest version and documentation can be obtained at:
% http://www.ctan.org/tex-archive/macros/latex/contrib/subfig/
% The latest version and documentation of caption.sty can be obtained at:
% http://www.ctan.org/tex-archive/macros/latex/contrib/caption/




% *** FLOAT PACKAGES ***
%
%\usepackage{fixltx2e}
% fixltx2e, the successor to the earlier fix2col.sty, was written by
% Frank Mittelbach and David Carlisle. This package corrects a few problems
% in the LaTeX2e kernel, the most notable of which is that in current
% LaTeX2e releases, the ordering of single and double column floats is not
% guaranteed to be preserved. Thus, an unpatched LaTeX2e can allow a
% single column figure to be placed prior to an earlier double column
% figure. The latest version and documentation can be found at:
% http://www.ctan.org/tex-archive/macros/latex/base/



%\usepackage{stfloats}
% stfloats.sty was written by Sigitas Tolusis. This package gives LaTeX2e
% the ability to do double column floats at the bottom of the page as well
% as the top. (e.g., "\begin{figure*}[!b]" is not normally possible in
% LaTeX2e). It also provides a command:
%\fnbelowfloat
% to enable the placement of footnotes below bottom floats (the standard
% LaTeX2e kernel puts them above bottom floats). This is an invasive package
% which rewrites many portions of the LaTeX2e float routines. It may not work
% with other packages that modify the LaTeX2e float routines. The latest
% version and documentation can be obtained at:
% http://www.ctan.org/tex-archive/macros/latex/contrib/sttools/
% Documentation is contained in the stfloats.sty comments as well as in the
% presfull.pdf file. Do not use the stfloats baselinefloat ability as IEEE
% does not allow \baselineskip to stretch. Authors submitting work to the
% IEEE should note that IEEE rarely uses double column equations and
% that authors should try to avoid such use. Do not be tempted to use the
% cuted.sty or midfloat.sty packages (also by Sigitas Tolusis) as IEEE does
% not format its papers in such ways.





% *** PDF, URL AND HYPERLINK PACKAGES ***
%
\usepackage{url}
% url.sty was written by Donald Arseneau. It provides better support for
% handling and breaking URLs. url.sty is already installed on most LaTeX
% systems. The latest version can be obtained at:
% http://www.ctan.org/tex-archive/macros/latex/contrib/misc/
% Read the url.sty source comments for usage information. Basically,
% \url{my_url_here}.





% *** Do not adjust lengths that control margins, column widths, etc. ***
% *** Do not use packages that alter fonts (such as pslatex).         ***
% There should be no need to do such things with IEEEtran.cls V1.6 and later.
% (Unless specifically asked to do so by the journal or conference you plan
% to submit to, of course. )

\newcommand{\Ivec}[1]{\mbox{\boldmath $#1$}}
\newcommand{\argmin}{\operatornamewithlimits{argmin}}
\newcommand{\argmax}{\operatornamewithlimits{argmax}}
\newcommand{\sgn}{\operatorname{{\mathrm sgn}}}
\newcommand{\mean}{\operatornamewithlimits{mean}}
\newcommand{\Bin}{\operatorname{{\mathrm Bin}}}
\newcommand{\Beta}{\operatorname{{\mathrm Beta}}}
\newcommand{\Gammadist}{\operatorname{{\mathrm Gamma}}}
\newcommand{\Uniform}{\operatorname{{\mathrm Uniform}}}

% correct bad hyphenation here
\hyphenation{op-tical net-works semi-conduc-tor}

%% % Squeeze space to fit 4 pages constraint:
%% % \linespread{0.97}
%% % or
\setlength{\parskip}{0pt}
\setlength{\parsep}{0pt}
\setlength{\headsep}{0pt}
\setlength{\topskip}{0pt}
\setlength{\topmargin}{0pt}
\setlength{\topsep}{0pt}
\setlength{\partopsep}{0pt}
% \linespread{0.982}
\linespread{0.976}

% \usepackage{hyperref}

\begin{document}
%
% paper title
% can use linebreaks \\ within to get better formatting as desired
\title{Fast Clustering for Interactive Tractography Segmentation}

% author names and affiliations
% use a multiple column layout for up to two different
% affiliations

% \author{\IEEEauthorblockN{Authors Name/s per 1st Affiliation (Author)}
% \IEEEauthorblockA{line 1 (of Affiliation): dept. name of organization\\
% line 2: name of organization, acronyms acceptable\\
% line 3: City, Country\\
% line 4: Email: name@xyz.com}
% \and
% \IEEEauthorblockN{Authors Name/s per 2nd Affiliation (Author)}
% \IEEEauthorblockA{line 1 (of Affiliation): dept. name of organization\\
% line 2: name of organization, acronyms acceptable\\
% line 3: City, Country\\
% line 4: Email: name@xyz.com}
% }

% \author{\IEEEauthorblockN{Emanuele
%     Olivetti\IEEEauthorrefmark{1}\IEEEauthorrefmark{2},
%     Thien Bao Nguyen\IEEEauthorrefmark{1}\IEEEauthorrefmark{2},
%     Eleftherios Garyfallidis\IEEEauthorrefmark{3},
%     Nivedita Agarwal\IEEEauthorrefmark{2}\IEEEauthorrefmark{4}\IEEEauthorrefmark{5}
%     and Paolo Avesani\IEEEauthorrefmark{1}\IEEEauthorrefmark{2}
%   }
%   \IEEEauthorblockA{\IEEEauthorrefmark{1}NeuroInformatics Laboratory (NILab),
%     Bruno Kessler Foundation, Trento, Italy}
%   \IEEEauthorblockA{\IEEEauthorrefmark{2}Centre for Mind and Brain
%     Sciences (CIMeC), University of Trento, Italy}
%   \IEEEauthorblockA{\IEEEauthorrefmark{3}Sherbrooke Connectivity
%     Imaging Laboratory (SCIL), Universit\'e de Sherbrooke, Canada}
%   \IEEEauthorblockA{\IEEEauthorrefmark{4}Radiology Unit, S.Maria del
%   Carmine Hospital, Azienda Provinciale dei Servizi Sanitari, Rovereto, Italy}
%   \IEEEauthorblockA{\IEEEauthorrefmark{5}Department of Neurology and
%     Department of Radiology, Section of Neuroradiology, University of
%     Utah, US}
% }


% conference papers do not typically use \thanks and this command
% is locked out in conference mode. If really needed, such as for
% the acknowledgment of grants, issue a \IEEEoverridecommandlockouts
% after \documentclass

% for over three affiliations, or if they all won't fit within the width
% of the page, use this alternative format:
% 
%\author{\IEEEauthorblockN{Michael Shell\IEEEauthorrefmark{1},
%Homer Simpson\IEEEauthorrefmark{2},
%James Kirk\IEEEauthorrefmark{3}, 
%Montgomery Scott\IEEEauthorrefmark{3} and
%Eldon Tyrell\IEEEauthorrefmark{4}}
%\IEEEauthorblockA{\IEEEauthorrefmark{1}School of Electrical and Computer Engineering\\
%Georgia Institute of Technology,
%Atlanta, Georgia 30332--0250\\ Email: see http://www.michaelshell.org/contact.html}
%\IEEEauthorblockA{\IEEEauthorrefmark{2}Twentieth Century Fox, Springfield, USA\\
%Email: homer@thesimpsons.com}
%\IEEEauthorblockA{\IEEEauthorrefmark{3}Starfleet Academy, San Francisco, California 96678-2391\\
%Telephone: (800) 555--1212, Fax: (888) 555--1212}
%\IEEEauthorblockA{\IEEEauthorrefmark{4}Tyrell Inc., 123 Replicant Street, Los Angeles, California 90210--4321}}




% use for special paper notices
%\IEEEspecialpapernotice{(Invited Paper)}




% make the title area
\maketitle


\begin{abstract}
  We developed a novel interactive system for human brain tractography
  segmentation to assist neuroanatomists in identifying white matter
  anatomical structures of interest from diffusion magnetic resonance
  imaging (dMRI) data. The difficulty in segmenting and navigating
  tractographies lies in the very large number of reconstructed
  neuronal pathways, i.e. the streamlines, which are in the
  order of hundreds of thousands with modern dMRI techniques. The
  novelty of our system resides in presenting the user a clustered
  version of the tractography in which she selects some of the
  clusters to identify a superset of the streamlines of interest. This
  superset is then re-clustered at a finer scale and again the user is
  requested to select the relevant clusters. The process of
  re-clustering and manual selection is iterated until the remaining
  streamlines faithfully represent the desired anatomical structure of
  interest. In this work we present a solution to solve the
  computational issue of clustering a large number of streamlines
  under the strict time constraints requested by the interactive
  use. The solution consists in embedding the streamlines into a
  Euclidean space and then in adopting a state-of-the art scalable
  implementation of the $k$-means algorithm. We tested the proposed
  system on tractographies from amyotrophic lateral sclerosis (ALS)
  patients and healthy subjects that we collected for a forthcoming
  study about the systematic differences between their corticospinal
  tracts.
\end{abstract}

%%% Local Variables: 
%%% mode: latex
%%% TeX-master: "olivetti_boi"
%%% End: 


\begin{IEEEkeywords}
diffusion MRI ; tractography ; clustering ; interactive segmentation 
\end{IEEEkeywords}


% For peer review papers, you can put extra information on the cover
% page as needed:
% \ifCLASSOPTIONpeerreview
% \begin{center} \bfseries EDICS Category: 3-BBND \end{center}
% \fi
%
% For peerreview papers, this IEEEtran command inserts a page break and
% creates the second title. It will be ignored for other modes.
\IEEEpeerreviewmaketitle


\section{Introduction}
\label{sec:introduction}

% What is dMRI and tractography
Diffusion magnetic resonance imaging
(dMRI)~\cite{basser1994mr} techniques provide
non-invasive images of the brain white matter. dMRI quantifies
locally, i.e. in each voxel, the diffusion process of the water
molecules which are mechanically constrained in their motion by the
axons of the neurons. Reconstruction and tracking
algorithms~\cite{mori2002fiber} transform dMRI data into a set of
streamlines, i.e. 3D polylines that approximate the neuronal
pathways. The whole set of streamlines is called tractography and it
represents the anatomical connectivity of the brain. See for example
Figure~\ref{fig}.

% % General clinical motivation for using dMRI
% REMOVE THIS PARAGRAPH. The investigation of white-matter anatomical
% structures plays a central role in understanding and characterising
% diseases such as multiple sclerosis and Alzheimer's disease. The
% hypothesis of a reduction in the anatomical connectivity is considered
% as one of the possible causes of them. The quantification of this
% reduction may be done from the analysis of tractography data and for
% this reason the segmentation of the tractography into known
% anatomically tract is a task of interest in neurological
% studies~\cite{clayden2009reproducibility}.

% Our specific motivation: tractography segmentation to study ALS
This work focuses on computer-assisted tractography segmentation and
describes the solution we developed to build a software system to
support neuroanatomists and medical doctors in studying the white
matter. Our work is motivated by a clinical research hypothesis about
the characterisation of the amiotrophic lateral sclerosis (ALS)
disease. We collected dMRI data from ALS patients and healthy controls
with the aim of studying the effects of the ALS disease on the
corticospinal tract (CST) (see Figure~\ref{fig}I), an anatomical
structure that connects cortical motor areas to the spine and the
body. The CST is known to be affected by
ALS~\cite{cosottini2010evaluation} and for this
reason, our long term goal is to characterise these effects through
tractography data. The first task in this endeavour is to segment the
CST from the full brain tractography of each subject.

% Brief description of the proposed computer-assisted segmentation process
Tractography segmentation can be performed manually or
automatically. Despite an increasing literature in automatic
segmentation (see a brief review
in~\cite{wang2011tractography}), the
application in the clinical domain usually rely on manual
segmentation. The manual segmentation process usually consists in
selecting the subset of the streamlines connecting a few manually
located regions of interest\footnote{See for example \url{http://www.trackvis.org}.}. This task is a lengthy and complex one,
for two reasons: first the tractography is a very large set of
streamlines, in the order of $3 \times 10^5$, which makes it
intrinsically difficult both to inspect and to unfold anatomical
structures (see Figure~\ref{fig}A). Second, the reconstruction of the
streamlines is frequently suboptimal due to the noise in the
measurement process and to the limitations of reconstruction
algorithms. For this reason, a single neuronal pathway may be just
partially reconstructed, resulting in multiple disconnected
polylines. These \emph{broken} streamlines would be discarded by the
manual segmentation procedure mentioned above, because not
connecting the regions of interest defined by the expert.

Differently from the previous approaches of tractography segmentation,
we conceived a novel computer-assisted interactive process based on
clustering algorithms, which aims at greatly reducing the time
required to manually segment a given anatomical white matter structure
of interest. 
% SHORT VERSION (SEE LONG VERSION BELOW):
Our approach is based on a fast-clustering technique by means of which
the expert is presented with a summary of the streamlines, i.e. the
clusters represented by their medoids~\footnote{A medoid is the
  element of a cluster closest to its centre.}. The expert manually
selects the medoids/clusters of interest in order to remove most of
the streamlines not related to the anatomical structure of interest.
Interacting with the summary, instead of the actual streamlines, is
much simpler for the user. In the example of Figure~\ref{fig}, the
user selects $15000$ streamlines (see \ref{fig}C) of the $3 \times
10^5$ streamlines (see \ref{fig}A) just by clicking on $20$ of the
$150$ medoids (see \ref{fig}B, the selected medoids are shown in
white).  The process of reclustering the selected streamlines and of
manual selection by the expert is iterated until the expert is
confident of having segmented the structure of interest (see
Figure~\ref{fig}I).
% THIS IS THE LONG VERSION OF THE PREVIOUS PARAGRAPH:
% Our approach is based on a fast-clustering technique by means of which
% the expert is initially presented a summary, i.e. the
% medoids~\footnote{A medoid is the element of a cluster closest to the
%   centroid.} of the clusters (see for example Figure~\ref{fig}B), of
% the whole tractography where she can manually select the subset of
% clusters to which to restrict the attention to. The selected subset of
% medoids (see them in white in Figure~\ref{fig}B) is an initial rough
% approximation the streamlines of interest (depicted in
% Figure~\ref{fig}C). This set is then re-clustered with an increased
% number of clusters so that more anatomical details are revealed (see
% Figure~\ref{fig}D). There the user can select what cluster is of
% interest and discard what is not. The selected clusters are then
% re-clustered in more and smaller clusters to reveal more details of
% the anatomical structure of interest. The selection and re-clustering
% process is repeated iteratively until the expert is confident of
% having segmented the structure of interest (see for example
% Figure~\ref{fig}I).

% Contents of the paper
In this work we describe the algorithmic solution we adopted in order
to build the interactive tractography segmentation tool. The core of
the problem is to cluster a large number of streamlines in no more
than a few seconds, to allow a comfortable interactive user experience
to the expert. The proposed solution combines two state-of-the-art
elements: first a recently proposed Euclidean embedding algorithm for
streamlines, i.e. the dissimilarity representation with the scalable
\emph{subset farthest first} (SFF) prototype selection
policy~\cite{olivetti2012approximation}. This embedding provides fast
and accurate vectorial representation of streamlines. Second, a
recently proposed improvement of the $k$-means clustering algorithm
called \emph{mini-batch} $k$-means~\cite{sculley2010web} (MBKM). This
algorithm, which require the data to lie in a vector space,
drastically reduces the convergence time to the actual clusters in
case of large and very-large sets of objects. We claim that the
dissimilarity embedding together with the MBKM algorithm provides a
viable solution to problem of fast clustering of streamlines.
% REMOVE
% We also tested a further potential
% improvement, i.e. a recently proposed initialisation algorithm for the
% $k$-means family, called $k$-means++~\cite{arthur2007kmeans}. But we
% excluded it from the final system due to poor performance as
% illustrated in Section~\ref{sec:experiments}.

% Structure of the paper
The paper is structured as follows. In Section~\ref{sec:methods} the
algorithmic elements of the proposed method are formally described. In
Section~\ref{sec:experiments} we describe the segmentation process and
report the details of the actual use of the proposed solution in the
context of the CST segmentation. We quantitatively describe the
segmentation process and provide timings to evaluate the viability of
the proposed solution. In Section~\ref{sec:discussion} we discuss the
results and we show that the proposed solution confirms our claims. 
% We conclude by mentioning the future steps for the development of our
% software and the open computational challenges that needs to be
% solved.

%%% Local Variables: 
%%% mode: latex
%%% TeX-master: "olivetti_boi"
%%% End: 


\section{Methods}
\label{sec:methods}
In the following we describe the elements that characterise our
proposed method. After introducing the notation we formally describe
the multiple scale representation and propose the method to choose these scales.

\subsection{Hierarchical clustering}
\label{subsec:hierarchical}
Given a set of input patterns denoted as $\mathcal{X} = \{\mathsf{x_1}, \ldots, \mathsf{x_j}, \ldots, \mathsf{x_N}\}$ where $\mathsf{x_j} = (x_{j1},x_{j2}, \ldots,x_{jd})^T \in \mathfrak{R}^d$ and each measure $x_{ji}$ is said to be a feature (attribute, dimension, or variable). Hierarchical clustering~\cite{johnson1967hierarchical} produces a structure of clusters of $\mathcal{X}$ that is more informative than the unstructured set of clusters returned by flat clustering. 
This characteristic meets the requirement of creating multiple scales of one original dataset $\mathcal{X}=\{x_1, \ldots, x_n\}$ 
%$\mathbb{P}$ of $\mathcal{T}$ 
%the immediately responding to the changing level of abstraction of user 
without re-running the clustering algorithm again. Another advantage is that hierarchical clustering does not require to pre-specify the number of clusters. It builds nested clusters by merging them successively, and this hierarchy of clusters represented as a tree (or dendrogram). The root of the tree is the unique cluster that gathers all the samples, the leaves being the clusters with only one sample. In another way, hierarchical clustering algorithm % (http://scikit-learn.org/stable/modules/clustering.html) 
can clusters data firstly on $n$ centers and consequently until only one center. This main character leads to the capability of visualizing $\mathcal{X}$ in many levels of abstraction, and the users can browse the value of level from $1$ to $N$, to see the clusters immediately.
%We refer \mathcal{H} as the whole hierarchical tree or dendogram of $\mathcal{X}$.

\begin{definition}
\label{def:hierarchical_tree}
A \textit{\textbf{hierarchical tree}} $\mathcal{H}$ of a N-object set $\mathcal{X}=\{\mathsf{x_1}, \ldots, \mathsf{x_j}, \ldots, \mathsf{x_N}\}$ is a collection of $Q$ partitions on $\mathcal{X}$: $\mathcal{H}=\{\mathsf{P}_0,\ldots,\mathsf{P}_Q\}$, with $Q \leq N$, such that $\mathsf{P}_0=\mathcal{X}$ and $C_{i} \in P_{m}, C_{j} \in P_{l}, m > l$ imply $C_{i} \subseteq C_{j}$ or $C_{i} \cap C_{j} = \emptyset$, for all $i,j \neq i, m, l = 1, \ldots, Q$.
\end{definition}
Every hierarchical tree $\mathcal{H}$ has a branch factor parameter $\gamma$ that quantifies how balanced the clusters are at any split.
Formally, $\gamma \geq \max_{intras C_i \in \{C_1, \ldots,C_k\}} \frac{max_i |C_i|}{min_i |C_i|}$
 where each intra $C-i$ is a non-leaf cluster
 %, partitioned into clusters ${C_1, \ldots, C_k}$. 
 $\gamma$ upper bounds the ratio between the largest and smallest clusters sizes across all intras in cluster $C$. This branch factor has been used in numerous analyses of clustering algorithms~\cite{eriksson2011active,balakrishnan2011noise}, and it is common to assume that the clustering is not too unbalanced.

Hierarchical clustering algorithms are either top down or bottom up. Bottom-up algorithms treat each streamline as a singleton cluster at the outset and then successively merge (or agglomerate) pairs of clusters until all clusters have been merged into a single cluster that contains all tracts. Bottom-up hierarchical clustering is therefore called Hierarchical Agglomerative Clustering (HAC). Top-down clustering requires a method for splitting a cluster. It proceeds by splitting clusters recursively until individual streamlines are reached \cite{johnson1967hierarchical}.
The basic process of HAC clustering~\cite{johnson1967hierarchical} is this:
\begin{enumerate}
	\item Assign each object to one cluster, and get the result of $N$ clusters, each of them contains just one streamline.
	%Let the distances (similarities) between the clusters the same as the distances (similarities) between the items they contain.
	\item Find the closest (most similar) pair of clusters and merge them into a single cluster.
	\item Compute distances (similarities) between the new cluster and each of the old clusters.
	\item Repeat steps $2$ and $3$ until all objects are clustered into a single cluster of size $n$.
\end{enumerate}

Step $3$ can be done in different ways, which distinguishes single-linkage, complete-linkage and average-linkage.
In \emph{single-linkage} clustering (also called the connectedness or minimum method), the distance between a pair of clusters $A$ and $B$ is the shortest distance from any streamline of one cluster to any streamline of the other cluster. 
%If the data consist of similarities, we consider the similarity between one cluster and another cluster to be equal to the greatest similarity from any member of one cluster to any member of the other cluster.
%${d}_sg(A,B) = \min_{i=1,\ldots,n_{s_A}} d({x}_i^A, s_B)$
\begin{equation}
\label{eq:distance_single_linkage}
d(A, B) = \min_{\mathsf{x}_A \in {A},\mathsf{x}_B \in {B}} d(\mathsf{x}_A,\mathsf{x}_B)
\end{equation}
where $d(\mathsf{x}_A,\mathsf{x}_B)$ is the distance between two objects
In \emph{complete-linkage} clustering (also called the diameter or maximum method), we consider the distance between cluster $A$ and cluster $B$ to be equal to the greatest distance from any member of one cluster to any member of the other cluster.
\begin{equation}
\label{eq:distance_complete_linkage}
d(A, B) = \max_{\mathsf{x}_A \in {A},\mathsf{x}_B \in {B}} d(\mathsf{x}_A,\mathsf{x}_B)
\end{equation}
In \emph{average-linkage} clustering, the distance between two clusters $A$ and $B$ is defined as the average distance from any streamline of cluster $A$ to any streamline of cluster $B$.
\begin{equation}
\label{eq:distance_average_linkage}
d(A, B) = avg_{\mathsf{x}_A \in {A},\mathsf{x}_B \in {B}} d(\mathsf{x}_A,\mathsf{x}_B)
\end{equation}
%\algsetup{indent=2em}
%%\newcommand{\factorial}{\ensuremath{\mbox{\sc Factorial}}}
%\begin{algorithm}[h!]
%\caption{Hierarchical clustering}\label{alg:hierarchical}
%\begin{algorithmic}[1]
%	\REQUIRE $\mathcal{X}=\{\mathsf{x_1}, \ldots, \mathsf{x_j}, \ldots, \mathsf{x_N}\}$
%	\ENSURE \textit{\textbf{hierarchical tree}} $\mathcal{H}$
%	%\REQUIRE An integer $n \geq 0$.
%	%\ENSURE The value of $n!$.
%	\medskip
%	\STATE $C_i \leftarrow \mathsf{x_i}, \forall i \in [1,\ldots, N]$
%	\vspace{1.5mm}
%	\STATE $\mathcal{H} \leftarrow \bigcup_{i=1}^{N}C_i$
%	\vspace{1.3mm}
%	\WHILE {$N > 1$}
%		\vspace{1.2mm}
%		\STATE $(C_a, C_b) \leftarrow agrmin(d(C_i,C_j)),$\\
%		\hspace{19mm} $ \forall C_i \in \mathcal{H},C_j \in \mathcal{H}, C_i \neq C_j$
%		\vspace{1.2mm}
%		\STATE $\mathcal{H} \leftarrow (C_a \cup C_b)$
%		\vspace{1.2mm}
%		\STATE $N \leftarrow N - 1$	
%	\ENDWHILE
%	\vspace{1.3mm}
%	\RETURN $\mathcal{H}$
%\end{algorithmic}
%\end{algorithm}
%After defining the distance measure between two clusters, the hieararchical procedure can easily done as the pseudo code in the algorithm~\ref{alg:hierarchical}.
\subsection{Multiple scales for visualization}
\label{subsec:multiple_scales}
The hierarchical tree $\mathcal{H}$ structures and presents dataset $\mathcal{X}$ at different levels of abstraction. A non-leaf cluster is composed of all its child clusters, while a leaf cluster contains only a single data item. The collection of all leaf-clusters presents exactly every data items $\mathsf{x}_i$ of $\mathcal{X}$, while the root is a cluster containing whole dataset $\mathcal{X}$ as one single node of the tree.
\begin{definition}
\label{def:level_detail}
Each cluster $C_i$ (node) of the tree $\mathcal{H}$, let $s(C_i)$ be the \textbf{\textit{level of detail}} of that cluster. This measurement $s(C_i)$ satisfies the following criteria: if $C_i$ is an ancestor of $C_j$, then $s(C_i) \geq s(C_j)$. 
\end{definition}
There are many properties of a cluster which could be used to measure $s(C_i)$. Among of them, two common use are the radius of a cluster (maximum distance between all pair samples of cluster $C_i$: $r_i = max_{\forall \mathsf{x}_a, \mathsf{x}_b
 \in C_i, \mathsf{x}_a \neq \mathsf{x}_b} \{d(\mathsf{x}_a,\mathsf{x}_b)\}$); and the hierarchical level of $C_i$ in the tree $\mathcal{H}$~\cite{yang2003interactive}.

\begin{definition}
\label{def:range_scale}
 The \textbf{\textit{range of scale}} of a hierarchical tree $\mathcal{H}$ is $[s_{min},s_{max}]$, where $s_{min} = \min_{\forall C_i \in \mathcal{H}}\{s(C_i)\}$, and $s_{max} = \max_{\forall C_i \in \mathcal{H}}\{s(C_i)\}$
\end{definition}
Depending on which property is used to measure the level of detail $s_i$, the value of  $s_{max}$ and  $s_{min}$ would be different. In the case of using the hierarchical level, let $h$ be the heigh of the tree $\mathcal{H}$, then the level of detail of cluster $C_i$: $s(C_i) = \frac{height(C_i)}{h}$, where $heigh(C_i)$ is the heigh of the cluster $C_i$. The scale range is from $[0,1]$, where $s_{min} = 0$ corresponds to the leaf with zero heigh, to $s_{max} = 1$ is at the root of the tree $\mathcal{H}$. However, in the case of using cluster radius, there is no guarantee that $s_{min} = 0$ and $s_{max} = 1$.

\begin{definition}
\label{def:cut}
 \textbf{\textit{A cut $\mathfrak{L}$}} of a hierarchical tree $\mathcal{H}$ at a given scale $w \in [s_{min},s_{max}] $  is $\mathfrak{L}(w)$:
\begin{equation}
\label{equ:a_cut_l}
\mathfrak{L}(w) = \{C_i | (s(C_i) \leq w  \wedge s(parent(C_i)) > w)\}
\end{equation}
where $parent(C_i)$ is the direct parent node of the cluster $C_i$
\end{definition}
Intuitively, the cut at $s_{min}$, $\mathfrak{L}(s_{min})$ is a set of all leaf clusters, while the $\mathfrak{L}(s_{max})$ is a single cluster representing the whole dataset $\mathcal{X}$. In general, $\mathfrak{L}(w)$ is a partion of $\mathcal{X}$, denoting a subset of the tree $\mathcal{H}$. $\mathfrak{L}(w)$ changes smoothly with the variance of the scale parameter $w$, which serves as the abstraction level of the dataset $\mathcal{X}$. It could be imagined that $\mathfrak{L}(w)$ is a cut across a vertically oriented hierarchical tree $\mathcal{H}$ that satisfies criteria: $\mathfrak{L}(w)$ intersecs each path of the tree $\mathcal{H}$, from the root to the leaf, only exactly at one point. The cutting point would depend on the value of parameter $w$. It should close to the root of the tree $\mathcal{H}$ when $w$ is high, and reversely. Moreover, the cut can be horizontal or unhorizontal (like zigzag) as long as for each path from the root to the leaf of the tree $\mathcal{H}$, there is only one crossing with $\mathfrak{L}(w)$. It is an open approach for cutting the tree comparing with the traditional one which only accepts the horizontal cut. 

\begin{definition}
\label{def:nested_in} 
Let $P$ and $Q$ be two partitions of dataset $\mathcal{X}$, $P = \{C_{1}^{P}, \ldots, C_{1}^{P}\}$ and $Q = \{C_{1}^{Q}, \ldots, C_{1}^{Q}\}$. Partition $P$ is \textbf{\textit{nested in}} partition $Q$, denoted as $P \preceq Q$, if and only if
\begin{equation}
\label{equ:nested_in}
P \preceq Q  \leftrightarrow \forall C_i^{Q} \in Q, \exists C^{P}_{i_{1}},\ldots, C^{P}_{i_{k}} \in P: C_i^Q = \cup_{t=1}^{k} C^{P}_{i_{t}}
\end{equation}
\end{definition}
\begin{definition}
\label{def:multi_scales_representation} Given the scale range $[s_{min},s_{max}]$ of a tree $\mathcal{H}$, the \textbf{\textit{multiple scales representing}} for the tree $\mathcal{H}$ is an \emph{ordered set} of $k$ scale values from $[s_{min},s_{max}]$: $\mathsf{\textit{B}} = \{b_1, b_2, \ldots, b_k\}, b_i \in [s_{min},s_{max}], \forall i \in [1,k]$, where $k$ is the order of set $\mathsf{\textit{B}}$, which satisfies the following condition:
\begin{equation}
\label{equ:multi_scales_represent}
\forall i \in [1, \ldots, k-1] : \mathfrak{L}(b_i)  \preceq \mathfrak{L}(b_{i+1})
\end{equation}
\end{definition}
It is usually that $b_1 = s_{min}$, where the whole elements of $\mathcal{X}$ are presented, and $b_k = s_{max}$, which coresponds to only one virtual representation of $\mathcal{X}$. However, the value of $k$ is still an open question and totally depends on the application. In the next section, we will discuss about how to define this value and also how to select each $b_i$ from $[s_{min},s_{max}]$.

\subsection{Choosing multiple scales for representation}
\label{subsec:choosing_multiple_scales}
In this part, we propose a simple and efficient a method to determine the good scales, highlighting meaning full clusters to represent for a hieararchical tree $\mathcal{H}$, which satisfies the condition in definition~\ref{def:multi_scales_representation}

Given a hierarchical tree $\mathcal{H}$, constructed from dataset $\mathcal{X}$, with the range scale $[s_{min},s_{max}]$, the problem is how to choose $\mathsf{\textit{B}} = \{b_1, b_2, \ldots, b_k\}, b_i \in [s_{min},s_{max}]$, $\forall i \in [1,k]$. With each cluster $C \in \mathcal{H}$, let $(\alpha_{min}^C,\alpha_{max}^C)$ be two scale factors which the cluster $C$ appears and disappears from the tree $\mathcal{H}$. Pascal et. al.~\cite{pons2011postprocessing} proposed a method to compute $\mathsf{\textit{B}}$ from a pairwise $(\alpha_{min}^C,\alpha_{max}^C)$. However, the procedure in~\cite{pons2011postprocessing} to calculate $(\alpha_{min}^C,\alpha_{max}^C)$ for each cluster $C \in \mathcal{H}$ is hightly complexity cost, while the hierarchical order of the tree $\mathcal{H}$ provides a good hint. By exploring this information, we susgest a more easy and efficient way: $\alpha_{min}^C = s(C)$ and $\alpha_{max}^C = s(parent(C))$, which is intuitively, as the definition of the level of detail in~\ref{def:level_detail}. Beside, it may consider that the good clusters would be present for a wide range of scale factors. Thus, the goodness of a cluster could be measured as $\alpha_{min}^C - \alpha_{max}^C)$ and the best scale representing $C$ as $\alpha = \frac{\alpha_{max}^C-\alpha_{min}^C}{2}$~\cite{pons2011postprocessing}.
\begin{definition}
\label{def:goodness_cluster} The \textbf{goodness function $R(C)$ of a cluster $C$} at a scale $w$ is:
\begin{equation}
\label{equ:goodness_cluster}
R_{w}(C) = \frac{\alpha_{max}^C-\alpha_{min}^C}{2} + \frac{2(\alpha_{max}^C- w)(w - \alpha_{min}^C)}{\alpha_{max}^C-\alpha_{min}^C}
\end{equation}
\end{definition}
\begin{definition}
\label{def:goodness_scale} Given a scale $w \in [s_{min},s_{max}]$, the \textbf{goodness fucntion $R(C)$ of a scale $w$} is:
\begin{equation}
\label{equ:goodness_scale}
R(w) = \frac{1}{N}\sum_{C \in \mathfrak{L}(w)} |C|R_{w}(C)
\end{equation}
\end{definition}
Obviously, $R(w)$ is a quadratic function of $w$, and ca be used for determining the scale factors corresponding to the good clusters. By focussing on the local maxima of $R(w)$, we can estimate good scales for representing the tree $\mathcal{H}$, and thus getting the $\mathsf{\textit{B}} = \{b_1, b_2, \ldots, b_k\}, b_i \in [s_{min},s_{max}], \forall i \in [1,k]$. Example of the $R(w)$ function can be found in the figure~\ref{fig:goodness_score}

\subsection{Evaluation the multiple scales for representation}
\label{subsec:evaluation_scales}
Based on the real fact that, while using our software tool to do the segmentation of cortinal spinal tract, medical practitioners usually choose $~15$ $(\lambda_1)$ clusters from $~50$ $(\lambda_2)$ representatives~\cite{olivetti2013fast}, we propose a method to evaluate the represented multi-scale set $\mathsf{\textit{B}}$ based on the \textit{split factor} as following.

\begin{definition}
\label{def:split_factor} \textbf{\textit{Split factor $\xi$}} of \textit{a cluster} $C \in \mathcal{H}$ to a scale $w \in [s_{min},s_{max}]$ is $\xi(C,s)$
\begin{equation}
\xi(C,s) = card(P(C,s))
\end{equation}
where $P(C,s) = \{ C_j | (C_j \in \mathcal{H}) \wedge (s(C_j) = w) \wedge (C_j \subseteq C) \}$ 
\end{definition}
\begin{definition}
\label{def:split_factor_set} \textbf{\textit{Split factor $\xi$}} of a \textit{set of cluster} $P=\{C_1, C_2,.., C_m\} \subseteq \mathcal{H}$ to a scale $s \in [s_{min},s_{max}]$ is $\xi(P,s)$
\begin{equation}
\xi(P,s) = \sum_{C_i \in P}\xi(C_i,s)
\end{equation}
\end{definition}
%Given a specific scale $w \in \mathsf{\textit{B}}$, draw uniformly at random $k_1$ clusters from the cut $\mathfrak{L}(w)$ of tree $\mathcal{H}$ at $w$, called $S_w$: $S_{w}=\{C_{s_1}, C_{s_2},.., C_{s_{k_1}}\}$, $C_{s_{i=1}^{k_1}} \in \mathfrak{L}(w)$.
\begin{definition}
\label{def:best_scales} The set of scales $\mathsf{\textit{B}} = \{b_1, b_2, \ldots, b_k\}$ is called \textbf{\textit{the best scales for representation}} of the tree $\mathcal{H}$, given $\lambda_1$ and $\lambda_2$, if the following condition satisfies
\begin{equation}
\label{equ:best_scale}
\forall b_i \in \mathsf{\textit{B}}: \lambda_2 - \Delta \leq \xi(S_{(b_i,\lambda_1)},b_{i-1}) \leq \lambda_2 + \Delta 
\end{equation}
where $S_{(b_i,\lambda_1)}$ is a Gaussian distribution subset of the cut $\mathcal{H}$ at scale $b_i$, $\mathfrak{L(b_i)}$, with the order of $\lambda_1$:
\begin{equation}
S_{(b_i,\lambda_1)} = \{C_1, \ldots, C_{\lambda_1}\}, C_j \in \mathfrak{L(b_i)}, \forall j \in [1, \ldots, \lambda_1] 
\end{equation}
\end{definition}
In the case of $b_1$, the split factor is computed to the leaf: $\xi(S_{(b_1,\lambda_1)},0)$. The pseudo code of it is presented in algorithm~\ref{alg:evaluate}
%the best cut scale set \mathsf{B}% = {b_1, b_2, \ldots, b_k}}
%\mathcal{H}
%\IncMargin{1em}
%\begin{algorithm}
	%\SetAlgoLined
%	\AlgoDisplayBlockMarkers\SetAlgoNoLine%
%	\SetKwData{Left}{left}\SetKwData{This}{this}\SetKwData{Up}{up}
%	\SetKwFunction{Union}{Union}\SetKwFunction{FindCompress}{FindCompress}
%	\SetKwInOut{Input}{input}\SetKwInOut{Output}{output}
%	\Input{a hierarchical tree $\mathcal{H}$\\
%	a set of cut scales $\textit{$\mathsf{B}$} = \{b_1, b_2, \ldots, b_k\}$}
%	\Output{accept \textit{$\mathsf{B}$} as a good represent for $\mathcal{H}$ or not}
%	\BlankLine
%	\tcc{initialization}
%	$accept \leftarrow True$\;
%	$i \leftarrow 1$\;
%	\While {$(i \leq k-1) \wedge (accept)$ }
%	{
%		$w \leftarrow b_i$\;
%
%		$l \leftarrow b_{i+1}$\;
%
%		%\tcc{draw $S=\{s_1,s_2,..,s_{k_1}\}$ of size $k_1$ uniformly at random}
%
%		$ S \leftarrow$ choose $\lambda_1$ clusters uniformly at random from $\mathfrak{L}$$(w)$\;
%
%		\tcc{calculate the split factor of $S$ to scale $l$}
%
%		$t \leftarrow \xi(S,l)$\;
%
%		\tcc{check the quality of the cut $b_i$ based on equation~\ref{equ:best_scale}}
%		\If{$(t \geq \lambda_2 + \Delta ) \vee (t \leq \lambda_2 - \Delta) $}
%		{
%			\tcc{update the result}
%			$accept \leftarrow False$\;
%		}		
%		$i \leftarrow i+1$\;
%	}
%	return $accept$\;
%\caption{Evaluate the set of cut scales, based on split factor}\label{algo_evaluate}
%\end{algorithm}
%\DecMargin{1em}
\algsetup{indent=2.5em}
\begin{algorithm}[h!]
\caption{Evaluate the set of cut scales, based on split factor}\label{alg:evaluate}
\begin{algorithmic}[1]
	\REQUIRE a hierarchical tree $\mathcal{H}$ \AND \\
             \hspace{8 mm} a set of cut scales $\textit{$\mathsf{B}$} = \{b_1, b_2, \ldots, b_k\}$
	\ENSURE accept \textit{$\mathsf{B}$} as a good represent for $\mathcal{H}$ or not
	\medskip
	\STATE $accept \leftarrow \TRUE$ \COMMENT{initialization}
	\STATE $i \leftarrow 1$
	\WHILE {$(i \leq k-1)$ \AND $(accept)$ }
		\vspace{1.4mm}
		\STATE $w \leftarrow b_i$

		\STATE $l \leftarrow b_{i+1}$
		\vspace{1.6mm}
		\STATE $ S \leftarrow$ choose $\lambda_1$ clusters uniformly at random \\
								\hspace{8mm} from $\mathfrak{L}$$(w)$
		\vspace{1.6mm}
		\STATE $t \leftarrow \xi(S,l)$ \COMMENT{split factor of $S$ to scale $l$}\\
		\vspace{1.6mm}
		\COMMENT{check quality of the cut $b_i$ based on equation~\ref{equ:best_scale}}
		\IF{$(t \geq \lambda_2 + \Delta )$ \AND $(t \leq \lambda_2 - \Delta) $}
			\vspace{1.3mm}	
			\STATE $accept \leftarrow \FALSE$ \COMMENT{update the result}
		\ENDIF
		\vspace{1.5mm}
		\STATE $i \leftarrow i+1$\;	
	\ENDWHILE
	\vspace{1.5mm}
	\RETURN $accept$
\end{algorithmic}
\end{algorithm}


\section{Experiments}
\label{sec:experiments}
TODO

Report timings for the actual dissimilarity projection of the entire
tractography and the timing of SFF.

Table: timing to cluster a given set of streamlines, i.e. at different
steps of the iterative refinement: 300K, 20K, 5K, 1K...

\subsection{ALS Dataset}
Brief description of the dataset.

\subsection{Example}
Sequence of pictures from the initial tractography to the segmented
cortico-spinal tract (CTS).

Another image to compare a patient's CTS vs. a healthy subject CTS.

Add a note to mention about the average amount of time to segment a
CTS in a given subject (5 minutes?).

Add a table about the number of clusters (selected/shown) and the
related number of streamlines addressed, at each iteration of the
segmentation process. Like (3/300)->(5000/300000),
(10/15)->(4000/5000), etc.

Mention DiPy and related URLs.

Mention this is an open-source project.


%%% Local Variables: 
%%% mode: latex
%%% TeX-master: "olivetti_boi"
%%% End: 



\section{Conclusion}
\label{sec:discussion}
We created a software tool to support the interactive segmentation of
tractography data with pattern recognition algorithms. In order to
handle the computational burden of clustering a large number of
streamline under strong time constraints, we proposed a solution based
on the dissimilarity representation and the MBKM algorithm. As shown
in Table~\ref{tab:results} (4th column) the time required to cluster
the streamlines with the proposed solution was always the lowest and
always $<1$s during interactive use, thus meeting the requirements for
a comfortable user experience. Conversely, the time required by the
standard $k$-means algorithm was inadequate (see the 3rd column in
Table~\ref{tab:results}).
% \item At the first step of the segmentation session the clustering of
%   the whole tractography requires $\approx 20$s with the proposed
%   method. This may be an issue with interactive use, but can be solved
%   by pre-computing this clustering once and then by storing the result
%   together with the actual dataset for future use.
As future work we plan to investigate further pattern recognition
algorithms to better support the expert during tractography
segmentation. 
% Two examples are: supervised segmentation and spatial queries.
To conclude we plan to improve the software segmentation tool in order
to make it production stable in near future.

%%% Local Variables: 
%%% mode: latex
%%% TeX-master: "olivetti_boi"
%%% End: 


\section{Acknowledgment}
*** ANONYMISED ***
% The authors are grateful to Prof.Mark B. Bromberg who helped with
% patient recruitment and Prof. Lubdha Shah and Prof.Perry Renshaw for
% their scientific advise in data acqusition. Many thanks to Beka Huber
% for her passionate work as research assistant.



% \input{appendix}

% conference papers do not normally have an appendix


% use section* for acknowledgement
% \section*{Acknowledgment}


\bibliographystyle{IEEEtran}

\bibliography{group-prni2013_boi-7255}

% that's all folks
\end{document}


%%% Local Variables: 
%%% mode: latex
%%% TeX-master: "olivetti_boi"
%%% End: 
