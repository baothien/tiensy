\begin{abstract}
  We developed a novel interactive system for human brain tractography
  segmentation to assist neuroanatomists in identifying white matter
  anatomical structures of interest from diffusion magnetic resonance
  imaging (dMRI) data. The difficulty in segmenting and navigating
  tractographies lies in the very large number of reconstructed
  neuronal pathways, i.e. the streamlines, which are in the
  order of hundreds of thousands with modern dMRI techniques. The
  novelty of our system resides in presenting the user a clustered
  version of the tractography in which she selects some of the
  clusters to identify a superset of the streamlines of interest. This
  superset is then re-clustered at a finer scale and again the user is
  requested to select the relevant clusters. The process of
  re-clustering and manual selection is iterated until the remaining
  streamlines faithfully represent the desired anatomical structure of
  interest. In this work we present a solution to solve the
  computational issue of clustering a large number of streamlines
  under the strict time constraints requested by the interactive
  use. The solution consists in embedding the streamlines into a
  Euclidean space and then in adopting a state-of-the art scalable
  implementation of the $k$-means algorithm. We tested the proposed
  system on tractographies from amyotrophic lateral sclerosis (ALS)
  patients and healthy subjects that we collected for a forthcoming
  study about the systematic differences between their corticospinal
  tracts.
\end{abstract}

%%% Local Variables: 
%%% mode: latex
%%% TeX-master: "olivetti_boi"
%%% End: 
