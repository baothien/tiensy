\begin{abstract}
  We developed a novel interactive system for human brain tractography
  segmentation to assist neuroanatomists in identifying white matter
  anatomical structures of interest from diffusion magnetic resonance
  imaging data (dMRI). The difficulty in segmenting and navigating
  tractographies lies in the very large number of \emph{streamlines},
  in the order of hundreds of thousands, that modern MR techniques
  provide. The novelty of our system resides in presenting a clustered
  version of the tractography to the user which interactively selects
  the clusters of interest to narrow down the initial tractography to
  a subset containing the anatomical structure. The set of streamlines
  of the selected clusters are then re-clustered in smaller clusters
  so that the user can refine the previous selection and then
  re-cluster again at a finer scale. This process is iterated until
  the union of the remaining clusters faithfully represent the desired
  structure. The proposed system adopts a solution to solve the
  computational issue of clustering a large number of streamlines
  under the strict time constraints requested by the interactive
  use. The solution consists in projecting the streamlines into a
  Euclidean space and then in adopting a state-of-the art scalable
  implementation of the $k$-means algorithm. We tested the proposed
  system on tractographies from amyotrophic lateral sclerosis (ALS)
  patients and healthy subjects to study the systematic differences
  within the corticospinal tract.
\end{abstract}

%%% Local Variables: 
%%% mode: latex
%%% TeX-master: "olivetti_boi"
%%% End: 
