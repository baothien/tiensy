\section{Conclusion}
\label{sec:discussion}
\begin{itemize}
\item We created an effective software tool to support the interactive
  segmentation of tractography data with pattern recognition
  algorithms.
\item To handle the computational burden of clustering a large number
  of streamline under strong time constraints, we propose a solution
  based on the dissimilarity representation and the MBKM algorithm.
\item As shown in Table~\ref{tab:results} the time required to cluster
  the streamlines with the proposed solution is always the lowest and
  always $<1$s, which meets the requirements of interactive use.
\item Conversely, the time required by the standard $k$-means
  algorithm becomes inadequate, i.e. $>5$s, with more than $\approx
  5000$ streamlines (see the 3rd column in Table~\ref{tab:results}).
\item At the first step of the segmentation session the clustering of
  the whole tractography requires $\approx 20$s with the proposed
  method. This may be an issue with interactive use, but can be solved
  by pre-computing this clustering once and then by storing the result
  together with the actual dataset for future use.
\item We also tested the $k$-means++ seeding to initialise the
  $k$-means and MBKM algorithm but observed slightly increased
  computational cost, as reported in Table~\ref{tab:results} (cf. 3rd
  vs 5th and 4th vs 6th columns).
\item To conclude we plan to improve the software segmentation tool to
  make it production stable.
\end{itemize}

%%% Local Variables: 
%%% mode: latex
%%% TeX-master: "olivetti_boi"
%%% End: 
