\section{Conclusion}
\label{sec:discussion}
We created a software tool to support the interactive segmentation of
tractography data with pattern recognition algorithms. In order to
handle the computational burden of clustering a large number of
streamline under strong time constraints, we proposed a solution based
on the dissimilarity representation and the MBKM algorithm. As shown
in Table~\ref{tab:results} (4th column) the time required to cluster
the streamlines with the proposed solution was always the lowest and
always $<1$s during interactive use, thus meeting the requirements for
a comfortable user experience. Conversely, the time required by the
standard $k$-means algorithm was inadequate (see the 3rd column in
Table~\ref{tab:results}).
% \item At the first step of the segmentation session the clustering of
%   the whole tractography requires $\approx 20$s with the proposed
%   method. This may be an issue with interactive use, but can be solved
%   by pre-computing this clustering once and then by storing the result
%   together with the actual dataset for future use.
As future work we plan to investigate further pattern recognition
algorithms to better support the expert during tractography
segmentation. 
% Two examples are: supervised segmentation and spatial queries.
To conclude we plan to improve the software segmentation tool in order
to make it production stable in near future.

%%% Local Variables: 
%%% mode: latex
%%% TeX-master: "olivetti_boi"
%%% End: 
