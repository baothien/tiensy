\section{Experiments}
\label{sec:experiments}
In the following we briefly describe the dataset we collected for the
validation of the proposed approach. Then we describe the details of
the interactive segmentation process on those data which is depicted
in Figure\ref{fig:} and conclude with the actual timing of the
competing approaches.

\subsection{ALS Dataset}
%BAO/PAOLO: Brief description of the dataset, i.e. the parameters of the acquisition, reconstruction algorithms, number of subjects etc.. 
The data is recorded with a $3T$ scanner at Utah Brain Institute. It consisted the recordings of $12$ ALS patients and $12$ healthy controls; $64$  ($+1$, i.e. $b=0$) gradients; $b$-value$=1000$.; anatomical scan ($1 \times 1 \times 1mm^3$).
We reconstruct the streamlines using EuDX, a deterministic tracking algorithm ~\cite{garyfallidis2012towards} from the DiPy
library~\footnote{\url{http://www.dipy.org}}.
We just present results on one subject.

\subsection{The Interactive Segmentation Process}
Describe in more detail the clustering-based interactive segmentation
process.

Sequence of pictures from the initial tractography to the segmented
cortico-spinal tract (CTS).


Report timings for the actual dissimilarity projection of the entire
tractography and the timing of SFF.

Add a note to mention about the average amount of time to segment a
CTS in a given subject (5 minutes?).


\begin{table}
  \centering
  \begin{tabular}{ r | r | c | c | c | c}
    size & $k$ & $k$-means(rnd) & MBKM(rnd) & $k$-means($k$++) &  MBKM($k$++) \\
    \hline
    \hline
    $500$    &  $50$ &  $0.3s$ &  $\mathbf{0.2}s$ &   0.5s  &  0.3s \\
    \hline
    $1000$   &  $50$ &  $0.6s$ &  $\mathbf{0.2}s$ &   1.0s  &  0.3s \\
    \hline
    $5000$   &  $50$ &  $6.1s$ &  $\mathbf{0.4}s$ &   7.4s  &  0.5s \\
    \hline
    $10000$  &  $50$ & $14.4s$ &  $\mathbf{0.6}s$ &  16.9s  &  1.0s \\
    \hline
    $15000$  &  $50$ & $29.9s$ &  $\mathbf{0.7}s$ &  32.4s  &  1.1s \\
    \hline
    $250000$ & $150$ & $>1000s$ & $\mathbf{21.8}s$ &  $>1000s$  &  24.8s \\
    \hline
  \end{tabular}
  \caption{Clusterning results.}
  \label{tab:results}
\end{table}



Mention DiPy, Scikit-learn and related URLs.

Mention this is an open-source project.


%%% Local Variables: 
%%% mode: latex
%%% TeX-master: "olivetti_boi"
%%% End: 

